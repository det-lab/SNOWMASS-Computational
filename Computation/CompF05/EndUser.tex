% !TeX root = ../../SnowmassBook-ComputationExperimental.tex

\setcounter{chapter}{4} 

%% IMPORTANT:   from this file, refer to the bibliography as              
%                                                          Computation/CompF05/bibliography.tex   
%%    refer to a figure   A.pdf  as    Computation/CompF05/figures/A.pdf  .

\chapter{End User Analysis for Particle Physics Computation}

\authorlist{G.~S.~Davies, P.~Onyisi, A.~Roberts}
   {(contributors from the community)}

\section{Introduction}
What do we talk about when we talk about analysis?


\section{People do software work} 

\subsection{Problems}
\begin{itemize}
    \item Lack of long-term support for software efforts (grant cycle is three years)
    \item Software efforts are siloed by collaboration - it's rare for people to be funded for cross-experiment software efforts.  E.g. if Peter writes a package that another collaboration wants to use, it is not a common pattern for that collaboration to fund him for that support.  Need mechanisms for people to get partial support to work beyond their immediate use cases.  And also for a field to say, ``yes, this software is important!"
    \item Recognition that supports stable careers for sustainable software work
    \item Misalignment between what we need for good software and what the field recognizes as valid work
    \item What else?
\end{itemize}


\subsection{Personnel support case studies}
\paragraph{PDG}
The PDG is recognized as important despite not being innovative.  We all expect the PDG to be there.

\paragraph{ROOT}
They don't have funding cycles?

\paragraph{XSEDE ECSS}

\paragraph{SGCI}

\subsection{Findings and Recommendations}
A lot of transformative ideas are introduced to the HEP computing community and are implemented by early-career scientists (especially grad students and postdocs). Support for these physicists can be minimal, and the career trajectories are opaque.\\
\textit{Recommendation.} Software work (especially with cross-experiment application) should receive stronger consideration for funding. More cross-experiment/frontier computing physicist positions could be created.  Funding agencies and frontiers need to work together to identify viable long-term funding patterns for this work.

% \section{Hardware Needs - Peter}
% \section{Hardware Needs}

% \begin{table}
%     \begin{center}
%     \caption{\label{tbl:analysis_hw}Fraction of overall CPU and disk budget allocated to analysis activities, as projected by various experiments.}
% \begin{tabular}{ccc}
%     \hline\hline
%     Experiment & CPU & Disk\\
%     \hline
%     ATLAS, HL-LHC \cite{ATLAS:2020pnm,Collaboration:2802918} & 7--8\% & 43--49\%\\
%     \hline\hline
% \end{tabular}
% \end{center}
% \end{table}
% Let CompF4 handle this?

\section{Analysis Ecosystems: Libraries, Languages, and Data Formats - Peter}
No analysis software functions entirely on its own; any package is situated in the context of the input data it consumes, the output data it produces, the other software it depends on, and the way it is configured or embedded in other code. Because of this we talk about ``software ecosystems,'' groups of packages which are typically used together.

Packages in an ecosystem typically have common data interchange formats and similar programming language interfaces. In some cases they may be distributed together as a single metapackage with overall versioning. There are two major ecosystems in HEP:
\begin{itemize}
    \item \textbf{ROOT} \cite{Brun:1997pa}: hosted by CERN, the ROOT suite is a tightly-integrated set of libraries that cover a broad range of HEP analysis needs, including I/O, event loop execution (including a parallel distributed mode), histogramming, fitting and statistical analysis, and visualization. The libraries are written in C++ and that is still considered the primary language for its use, although the Python bindings (PyROOT) are very well supported and by construction expose essentially the full API.  Bindings for the R language are also currently supported. The ROOT libraries were developed to meet the specific needs of HEP experiments and as such provide solutions that are well-matched to HEP analysis problems, although this also means that use outside of HEP is limited. The integrated, tightly-bound nature of ROOT means that using alternative software for any particular functionality can be difficult. ROOT is undergoing a redesign \cite{Naumann:2022pub} (``ROOT 7'') which aims to improve interfaces with the benefit of modern C++ and the benefit of over 25 years of practical experience. 
    \item \textbf{Python}: this is a somewhat loose term for a set of tools, with Python as the primary language interface, introduced with the primary goal of enabling the use of software developed outside of HEP, in particular for machine learning. Many core packages in this ecosystem are part of the Scikit-HEP \cite{Rodrigues:2020syo} and IRIS-HEP \cite{IRISHEPWEB} projects. This ecosystem is still in development and has no single governance team. It tends to emphasize independent packages for different aspects of the analysis pipeline (so, for example, I/O is handled with a different package from histogramming). Due to a standard and extensive set of tools aimed at supporting development of open source Python packages (the pypi package repository, github actions for continuous integration, documentation hosting on readthedocs) the barriers to entry for new software in this ecosystem are quite low. Development teams in this ecosystem are typically small and feature junior personnel. 
\end{itemize}

The long history of ROOT in HEP means that similar featuresets have sometimes been developed more than once, in incompatible fashions. For example, parallel processing of events on a single node was supported by PROOF Lite, the implicit multithreading feature for executing queries on TTrees, and RDataFrame. In the Python ecosystem a similar situation arises due to simultaneous development and shorter development cycles.

Ecosystems associated with other programming languages, notably Java \cite{Chekanov:2020bja}, Go \cite{Binet:2018xcc}, and Julia, have been developed and seen use in certain situations. Adoption of these is coupled to use of the relevant languages in the community, which at the moment is minimal.

\subsection{Programming Languages}
Users interact with software libraries and packages through programming languages. These can be separated into general-purpose languages (GPLs) which can be used for any task, and domain-specific languages (DSLs) (ROOT TCut, PAW kumac, etc.) which provide a restricted set of higher-level primitives which simplify certain operations. The two most commonly-used general-purpose languages in HEP are C++ and Python. Examples of domain-specific languages are the TCut syntax used for applying selections and constructing new variables in ROOT, and the kumac language used to control the old FORTRAN-based PAW suite.

General-purpose languages, by definition, are extremely capable and are used to solve problems outside of HEP. Exposure to these languages is one of the major technical skills that is transferable outside the field. It is considered necessary for HEP students to develop familiarity with, and preferably proficiency in, at least one general-purpose language. It is not necessarily the case that the languages that are used in HEP are the ones prevalent in the industries that particle physicists transition to (for example, R is widely-used in data science and virtually unknown in HEP). Because of their complexity, the time it takes to train personnel in them, and the need to transfer responsibilities for maintaining code from one person to another, the diversity of general-purpose languages used in the field is strictly limited. By contrast, domain-specific languages historically have been easier to master due to the limited range of constructs available.

HEP has had relatively few general-purpose languages in recent history. Until the early 2000s FORTRAN was commonplace. A desire to move to more modern and commonly-used languages drove a transition across the field to C++ (although there was competition, notably from Java); this was generally a top-down move imposed as new experiments were built or experimental upgrades were implemented. Often experiments found it valuable to use a second general-purpose language such as Python or tcl as a high-level scripting system for their data processing code. Python in particular came to be adopted by users as a convenient language and its simultaneous adoption as the standard language in machine learning has driven massive bottom-up adoption of the language. Availability of library bindings in various languages is extremely important for their use; both C++ and Python raise significant barriers to using libraries written in those languages elsewhere, although thanks to a lot of work the border between those two specific languages is relatively low.

Python is not necessarily an optimal language for scientific computing. In its reference implementation, it is a fully interpreted language, making it much slower than C++ for many tasks. The speed issue creates a programming paradigm in which users express operations via intermediate libraries (such as numpy for data manipulation or TensorFlow for neural network construction), introducing what amount to mini-languages embedded in Python. For this reason there is interest in exploring languages that can combine the expressiveness and ease-of-use of Python with compilation to machine code; the most commonly-explored option is Julia \cite{Stanitzki:2020bnx}. However it is clear that introducing another general-purpose language in HEP will require a very compelling case and it appears that the status quo regarding Python will continue for the foreseeable future, perhaps including the adoption of acceleration technologies such as numba \cite{10.1145/2833157.2833162}.

Visualizations are increasingly being done via web browsers (such as Jupyter notebooks or JSROOT). By far the dominant language in that environment is JavaScript, which is a language which particle physicists generally have very little experience in, and one where best practices have evolved very rapidly. If critial parts of the analysis ecosystem are written in this language, expertise will need to be maintained at some level.

Ideally it would be possible to seamlessly code written in different languages in a single application. This is extremely difficult to achieve in the general case due to languages' different memory layouts, calling conventions, and assumed invariants. Specific pairings, such as Python/C++, have solutions for interoperation. A general solution could involve leveraging languages' foreign function interfaces (FFI) through some common layer, such as the LLVM \cite{LLVM:CGO04} Intermediate Representation.

Domain-specific languages in HEP span a range of applicability, from steering reconstruction workflows \cite{Bennett:2022gyi}, to specifying variable construction for specific histograms to the construction of statistical analysis to a complete event selection workflow \cite{Prosper:2022lnf}. In a meaningful sense this includes the expressions needed to control numpy/AwkwardArray/TensorFlow from Python, or to express operations using the ROOT RDataFrame \cite{Piparo:2019xdy}; although formally these are library operations, analysts are expected to learn how to express their intent in terms of high-level operations while the details of execution are kept intentionally opaque. 

Domain-specific languages have the advantage of providing high-level primitives which in principle permit optimization of the actual execution of the code. In particular this includes parallelization and acceleration, operations which analysts may be uncomfortable with or which require significant investment to implement properly but which can provide speed and capability improvements when available. The major disadvantage is the (usually) restricted scope of operations that can be expressed in DSLs, which are typically designed with specific tasks in mind and which may make it unnatural or impossible for users to do other things. This is particularly dangerous for DSLs that operate at the ``analysis description'' level; it is not clear that such languages can generalize reasonably between frontiers.

Users can come to regard DSLs as primary parts of the analysis interface and the learning curve for them can be high. In particular, if there is more than one DSL relevant to a specific task, users may prefer to learn only one. If DSLs are linked to specific libraries or ecosystems, the synergies are liable to tie users to those technologies. 

\subsection{Data Formats}
Analysis data come in many forms:
\begin{itemize}
\item \textbf{Event data:} these consist of information, typically with a fixed but complex schema, describing individual events.
\item \textbf{Histograms:} These summarize features extracted from event data.
\item \textbf{Other summary data:} There are forms of summary data that cannot reasonably be expressed via histograms: sometimes tabular data is a better fit and more space-efficient, and sometimes the schema for the data is sufficiently complex that it makes sense to store a sui generis kind of object (such as for the results of fits).
\item \textbf{Configuration data:} The configuration for running some software may be stored in a human-readable and -editable format, or in a binary format --- the latter is especially common when one package is configuring the operation of another. In either case, interacting with the stored configuration requires data access, and it may be possible to alter the configuration like any other kind of data. 
\item \textbf{Metadata:} For end-user analysis, this tends to primarily be provenance tracking and version information.
\end{itemize}

File formats can describe both the overall container for data and the specific types of objects that can be stored. ROOT separates these fairly strictly, in that the ROOT file format can store essentially any C++ object, and ROOT objects can be serialized to formats other than ROOT files (such as JSON or XML). ROOT provides a number of pre-defined data objects, such as tables (known as TTrees) and histograms, and multiple objects can be present in the same file. Other data formats allow less freedom; for example, Apache Parquet merges the container and the data object and is only suitable for tabular data, while HDF5 files allow for a specific set of contained structures.

It is important to note that analysis end-users very rarely interact directly with underlying file formats --- they work with in the in-memory transient representation of data, rather than the persistent format, and the translation between the two is handled by libraries. The capabilities of specific formats may limit what users can do, and certain formats may provide more optimized storage, but otherwise the details are generally hidden from users. Therefore transitions in data format are easier to handle than those in libraries or languages. Newer versions of ROOT include the capability to read in data in CSV text format, sqlite files, or Apache Arrow.

\subsection{Visualization}

End-user analysis relies critically on visualization to give feedback to the physicist. Plots allow the user to quickly understand characteristics of the data but also to debug code and workflow problems.

Previous generations of analysis libraries supported interactive visualization through native graphics libraries; remote use involved the use of technologies such as remote X Windows. Over high latency links this can be extremely difficult to use and requires specific software to be installed on the user's machine. Recent trends exploit the near-universal availability of web browsers following common standards to offload rendering and interaction to the user's browser. This results in a more uniform experience across platforms and reduces external dependencies. This is the standard mode for code in Jupyter notebook environments (in particular the Python ecosystem) and is the baseline for future ROOT graphics.

The ROOT ecosystem's visualization libraries are naturally matched to the specific HEP use case. Visualization in the Python ecosystem is typically handled through the Python matplotlib package, a standard for scientific plot creation; matplotlib does not directly support a number of common HEP plot forms or histogram structure formats, so additional libraries have been written to help bridge these boundaries.

Although event displays are not typically used as part of an analysis workflow, they are still of interest to end users, and the same issues apply. Work is being done on experiment-agnostic event displays that render in browsers using JavaScript.

\subsection{Requirements for a Sustainable End-User Software Ecosystem}
A choice of software stacks is becoming available for HEP analysis. It is generally thought that the ROOT and Python stacks have different strengths and will competitively coexist into the future.

In order for the ecosystems to achieve long-term, sustainable success, we note the following requirements:
\begin{itemize}
    \item \textbf{Support of Personnel.} Unsupported software projects undergo ``code rot'' over time, a process where changes external to the package itself cause it to lose functionality. In the Python ecosystem, for example, the migration from Python 2 to Python 3 rendered old versions of packages unusable. For this reason alone, it is mandatory that any software that forms a key part of a HEP analysis ecosystem must have a maintainer who is able to provide the necessary level of support for the package. The community must understand that maintenance is a task of equivalent importance to developing new code, and recognize people's work appropriately. There must also be a mechanism for transferring responsibility for a package as necessary.
    \item \textbf{Documentation and Training.} Analysis software ecosystems are used as a gestalt --- not as a disconnected set of packages. 
    \item \textbf{Interoperability.} Ecosystem lock-in can be a problem for multiple reasons. Packages that can only be used in a single ecosystem will not provide benefits to users not in that ecosystem. 
\end{itemize}

\subsection{Findings and Recommendations}
There are two primary ecosystems developing for analysis in the next decade: one based on the ROOT libraries and one based on a collection of Python libraries. These projects have similar goals and scope but are organized differently and have different philosophies regarding industry tool use.\\
\textit{Recommendation.} Development of both ecosystems should be supported. The friendly competition between the two has already resulted in significant improvements for users. Maintaining interoperability between the two (e.g. in data formats) should be a priority.

The main general purpose languages used in HEP today are C++ and Python. Projects such as PyROOT have enabled interoperability of these languages at a level not replicated by other options in HEP.\\
\textit{Recommendation.} These two languages both have important roles to play in their respective niches and are expected to remain dominant in the near future. Other languages are generally unfamiliar to the community, have weak interoperability with existing libraries, and impose a maintenance burden if used without careful planning. Outside of bottom-up projects in other languages, or overwhelming domain-specific needs (e.g. JavaScript for visualization in web browsers), C++ and Python should be recommended. There is a reasonable concern about the efficiency of using interpreted languages in core analysis kernels, both in terms of computing resource use and consequent environmental impact. We recommend that the Python ecosystem community demonstrate that this issue has been addressed, and also ensure that just-in-time compiler technology (e.g. numba) is easy to integrate into user code.

A number of domain-specific languages have been proposed to address problems in various spheres, from specifying operations in columnar analysis, to specifying likelihood construction from histograms, to describing analysis at a very high level. Additionally, certain uses of Python (e.g. when scripting ML libraries, or working with RDataFrame) can be viewed as a DSL as they require library-specific knowledge.
\textit{Recommendation.} ``High-level'' DSLs that attempt to describe analysis via high-level objects are unlikely to generalize well between experiments as often the ``primitives'' are too different, but can be useful in certain situations. The lower-level DSLs can be extremely useful where relevant, however effort should be taken to avoid unnecessary duplication of scope as this imposes burdens on users similar to using multiple general purpose languages.

The ROOT container file format is ubiquitous in HEP, and comes with a serialization scheme tightly linked to the ROOT libraries. Industry and non-HEP tools typically use other formats (e.g. Apache Parquet for the ROOT TTree), and efforts are ongoing to enable their use. In addition the ROOT team foresees an evolution of the TTree to the more-optimized RNtuple. Multiple independent implementations of ROOT I/O are available. As ROOT is specialized to the HEP use case, it supports features missing from other formats.\\
\textit{Recommendation.} The ROOT file format is extremely important for ongoing experiments and historical data and compatibility must be maintained. Other formats have important use-cases. Tools to translate between formats, and to enable various ecosystems to ingest and produce them, should be maintained.

Software typically has low barriers to initial entry but significant ongoing maintenance requirements. New software projects are frequently initiated without much concern for long-term support (and this is reasonable since many bottom-up projects do not succeed).\\
\textit{Recommendation.} Computing personnel should be funded specifically to maintain software projects identified as satisfying an important need in the community (for example, by the HEP Software Foundation). 


\section{Analysis Models}
Once analysis code is written, it must be run on data. An analysis pipeline may involve multiple stages of data reduction and different codes, executing on very different platforms.

Analysis users value fast turn around --- being able to quickly answer physics questions. Because the rapidity with which analysis workflows can complete is paramount, data processing architectures which are well-suited to managed production workflows may not match well on to analysis tasks. Users often desire to run analyses on hardware that is under their control, and in fact such capability may be essential to allow users to develop, test, and debug their code. 

\subsection{Scale}
Users need to be able to scale code execution from a few events (for testing) to an experiment's full dataset (for actual analysis). The former requires interactive response, while the latter may require distributed execution (on a single cluster or across multiple sites). Interactive use historically has occurred via terminal sessions and visualization software native to the particular operating system and environment. Distributed execution has occurred on batch systems, either single-site or multi-site; in the latter case, especially if a federated computing model is adopted, complex issues of data locality and access, bookkeeping, job brokering, fair access, and so on arise.

Particle physics problems are usually \textit{high-throughput}, not \textit{high-performance}, problems: that is, they consist of a very large number of fairly lightweight computations which are essentially independent of each other, and so do not require computing resources to appear as a single, very powerful image (as on a traditional supercomputer). Traditional tightly-coupled supercomputer execution environments such as OpenMPI are therefore not typically needed for HEP analysis applications and in fact may be detrimental as they do not exploit the fine granularity of HEP problems. However, the increasing exploitation of coprocessors and accelerators (such as GPUs) in HEP code requires libraries that couple CPU and GPU execution on a single node.

Solutions that smoothly scale from small-scale interactive tests to full-data processing are desirable. In the absence of such solutions users face a barrier during the development of their analyses which can be quite substantial. Such solutions need to abstract away the execution of the event loop so that it can be executed on whatever resources are available, transparently to the user. Therefore by necessity they restrict the form of the user code to some extent. 


\subsection{Interfaces}

The ``traditional'' HEP analysis execution environment is a terminal. GUI applications add significantly more coding complexity and generally limited benefits, although they have been used (for example, in the ROOT PROOF suite, which still generally is invoked via a terminal). The use of a terminal allows terminal scripting languages, such as bash, to be used to orchestrate a workflow.

There is a recent trend towards using \textit{notebooks}, particularly those provided by the Jupyter environment, as the main interface for user code execution. Jupyter notebooks, which are rendered in a browser but with a connection to a backend kernel at the actual code execution site, function essentially as recorded interactive terminal sessions with additional documentation and output visualization capabilities. Jupyter notebooks are ill-suited for execution as actual code and are instead primarily used to script libraries which (for example) spawn worker tasks to actually perform the requested computations.

\subsection{Findings and Recommendations}
Users need to be able to scale their analyses from simple tests on a small debugging dataset to full deployment over all data. In many cases this transition requires working in a different environment (a common case is transitioning from a local workstation, referring to specific files, to a batch job running on a catalogued dataset). This can involve significant difficulties for users and cause support issues.\\
\textit{Recommendation.} There is unlikely to be a one-size-fits-all solution for all experiments. Recent work with interactive analysis facilities which provide extreme scaling capabilities to users through a single interface may address many of these issues; if successful the resulting software stacks should be made available to small experiments in a turnkey way.

\section{Dataset Bookkeeping and Formats - Amy}
\begin{comment}
S. V. Chekanov, G. Gavalian, and N. A. Graf, “Jas4pp - a Data-Analysis Framework for Physics and Detector Studies”, arXiv:2011.05329 [physics.comp-ph ]] (pdf).
 - Java-based programs make distribution and installation on Windows, Linux, Mac easy
 - They mention excellent library support but it's not clear what libraries?
 - based on JAIDA, the Java implementation of AIDA (Abstract Interfaces for Data Analysis)
 - supports  LCIO [21] I/O library developed for ILC studies. Some examples of reading LCIO files using Jython code can be found in Appendix A.2 (and in the following sections).
 - HiPO (High Performance Output data, from JLAB, has an XROOTD driver)
 - ProMC, ProIO
 - stores outputs with Java serialization method, can be binary format or XML 
 
Jim Pivarski, Eduardo Rodrigues, Kevin Pedro, Oksana Shadura, Benjamin Krikler, Graeme A. Stewart. ”HL-LHC Computing Review Stage 2, Common Software Projects: Data Science Tools for Analysis”, arXiv:2202.02194 [physics.data-an] (pdf).
- physicists are motivated to contribute
- interoperability is key to supporting scientists
- ROOT is the columnar data store that will always be with us.  But ROOT files might some day contain more than TTrees, in particular RNTuple is under active development
- Apache Arrow, Apache Parquet now offer similary-efficient columnar storage and tese format are used by some collaborations
- Why not databases, this is an obvious match to our access problems!  See Striped, ServiceX, SkyhookDM, Coffea's columnservice, Tiled

J. V. Bennett, J. Guilliams, M. Hernandez Villanueva, D. E. Jaffe, P. J. Laycock, A. Panta, C. Serfon, I. Ueda. ”Belle II grid-based user analysis”, arXiv:2203.07564 [hep-ex] (pdf).
- ROOT files for storage and analysis
- about 60 PB needed for all data (skimmed and simulation, not saving "original" data?)
- collaboration feels that this is feasible storage-wise (although large) but that 10^12 events represents a data management challenge
  -- Using Rucio for job submission and file resolution
  -- Concerns about scalability
- central question is: how do we fund what's needed for analysis after the experiment?
  -- storage, computing, and networking services all cost money.  In particular, grid solutions require security upkeep and therefore may not be feasible long-term
  -- software must remain usable (perhaps through containers)
  -- float the idea of central facilities that provide needed services to experiments

Harrison B. Prosper, Sezen Sekmen, Gokhan Unel. ”Analysis Description Language: A DSL for HEP Analysis”, arXiv:2203.09886 [hep-ph] (pdf). (also relevant to CompF07)
- emphasizes the need for semantic analysis to make collaboration with theory easier
- also makes an analysis self-documenting (Amy: to a degree)
- prefers a language-independent DSL, an "external DSL" for easier interoperability
- identifies DSL requirements as (1) easily understood by a physicist, (2) unambiguous, (3) domain complete.  I feel like (1) and possibly (2) need more clarification for this to be a true requirement list
- people do use the Analysis Description Language (analysis schools, Future Circular Collider studies), not clear if the use is limited to a specific group or groups, no complete LHC analysis yet
- Obvious enhancement of existing DSL would be to re-write to target LLVM.  (Amy: huh, okay, maybe it is time to think about DSLs?)

C. Backhouse. ”The CAFAna framework for neutrino analysis”, arXiv:2203.13768 [hep-ex] (pdf). (also under NF01)
- I feel like I need to re-read this paper (Amy)
- ROOT format files
- has been used by multiple neutrino experiments!
- connected with STAN to provide MCMC + analytic derivatives
\end{comment}

Data in the Cosmic Frontier falls into four categories, and each has an array of associated formats.  ROOT files are inescapable and present in virtually every type of analysis, although there are other formats in use:   

\begin{itemize}
  \item Raw data: overwhelmingly, custom binary formats that are specific to an experminent
  \item Processed and/or skimmed data: ROOT, always and forever, and we all know it
  \item Intermediate files that facilitate tool use: Parquet, Zarr, etc.  HDF5 is rarely used, its libraries are a bit painful and all the tools people want to use work with file formats that have friendlier IO libraries
  \item Metadata (file metadata, run information, processing states): Databases, stand-alone text files, wikis.  Metadata is also often included in the data file.
\end{itemize}

The challenges we face that are directly related to file formats are

\begin{itemize}
  \item Preservation.  Reproducibility, training, and open science/FAIR are increasingly of interest and importance to the community.  Making full use of our data requires documentation about the format, software that can read the format and provide analysis utility, and extensive metadata so that analyzers can make interpret or make use of the data.  Related efforts include FAIR, TOPS, and more.  There are significant international investments in these efforts; US-based efforts will require funding models beyond the traditional short-term grant cycle.  Preservation efforts must be guaranteed for at least 10 years to see community uptake.
  \item Interoperability. Re-using already-developed tools is an appealing way to reduce the software development burden that slows down experiments across the field.  To do this, data formats must be compatible with these tools.  The desire for interoperability has driven experiments to consider non-ROOT file formats like Parquet and HDF5.  This need has also created tools like uproot, which can read in ROOT files and present them in data formats that are compatible with common python libraries.
\end{itemize} 


\subsection{Data Cataloging and Delivery}

Users need to be able to obtain the input data for their analyses at the sites where their computation occurs. This means they, or the analysis infrastructure, must be able to discover where the data are stored and handle any data transfer/networking issues involved in bringing the data and compute resources together.

Data catalogs are useful even when the data are all stored in a single site, since they group related files into datasets and store useful, queryable metadata. When storage resources are distributed then catalogs become even more critical. Data placement is beyond the scope of this report but can be handled by systems that also provide a data catalog, such as Rucio \cite{Barisits:2019fyl}.

Data delivery to analysis tasks can be handled in a number of ways. On a small scale, the data may be provided via a POSIX filesystem (possibly networked). At a large scale, the data may need to be copied or streamed between locations over a wide area network. When multiple sites are involved, issues of authentication and identity management need to be addressed; maintaining multiple distinct accounts can be a heavy burden to users. Federated identity management and token-based authorization can address these issues.  Data transfer to a user job from remote storage can be handled in various ways; a common HEP standard is the xrootd protocol. Large-scale, distributed storage in industry is generally provided via object stores, for which the S3 protocol is the natural choice.

Standalone files that are considered opaque to data-handling and cataloging software, and only interpreted by end-user code, are not the only possible way to store and provide data. ``Intelligent'' infrastructure can provide a database-like view of datasets, enabling event selection and column reduction as a service (see e.g.\ \cite{Galewsky:2020xig}). User code need never interact with the ``native'' data format at all. In such an architecture the backend data store may itself be structured as a database \cite{Gutsche:2020kmd}. Work in this direction may be able to leverage advances in computational storage technology.

\subsection{Findings and Recommendations}
Experiments usually have global solutions for handling metadata involved in dataset cataloging and workflow control. However these solutions frequently are not available for individual analyses and users may need to develop bespoke solutions.\\
\textit{Recommendation.} Effort should be put into developing user-friendly data provenance and metadata storage systems that can be easily integrated into typical analysis tasks.

\section{Collaborative Software - Gavin}
\begin{comment}
T. Aarrestad et al. [HEP Software Foundation], “HL-LHC Computing Review: Common Tools and Community Software”, arXiv:2008.13636 [physics.comp-ph ]] (pdf).

Simone Campana, Alessandro Di Girolamo, Paul Laycock, Zach Marshall, Heidi Schellman, Graeme A Stewart. ”HEP computing collaborations for the challenges of the next decade”, arXiv:2203.07237 [physics.comp-ph] (pdf).

Dave Casper, Maria Elena Monzani, Benjamin Nachman, Costas Andreopoulos, Stephen Bailey, Deborah Bard, et al. ”Software and Computing for Small HEP Experiments“, arXiv:2203.07645 [hep-ex] (pdf). (also under EF0, NF0, RF0, CF0)

\end{comment}

At its heart, HEP research is collaborative in nature and as the complexity of the physics problems increases this necessitates the need to effectively and efficiently communicate and collaborate with peers.
The size and available expertise of an experiment may dictate what tools can be employed. When talking about collaborative software one is considering code management, version control, communication software, forums, wikis, Q\&A platforms.
%Reference the community survey
\subsection{Tools for Collaboration}

\begin{itemize} 
  \item Code Management and Version Control
In terms of code management and version control, many people love git. Equally many feel that more git training is needed. However, some experiments also support subversion and CVS (the Concurrent Versioning System), in some cases for legacy code, and so expertise and exposure is still necessitated
within experiments for these additional alternatives. Effort should be made to establish the use of a version control platform at the early stage of a user's research career. 
  \item Communication
In order to stay informed and in contact with collaborators all users use some form of instant messaging platform for both synchronous and asynchronous communication. Video and voice conferencing is dominated by the uptake of Zoom but other formats exsit and are still used, such as Skype, Vidyo). Email and phone communication are fast being overtaken as the primary method of communication by several platforms that offer IRC-style 
features, persistent channels (or chat 'rooms') organized by topic, private groups, and simple person-to-person direct messaging. Examples of such messaging programs include Slack, Zulip, Mattermost, Discord, Discource. 
DO these platforms meet our needs? Should we seek better integration? Many offer integration with various communication tools to be able to switch between (for example, Slack integration with GitHub, Zoom, Google tools) but often at a premium, per-user cost basis (ablbeit with education/research based cost reduction). Indeed programs such as Zulip are open-source and afford 
more control and zero financial burden at the cost of humanpower and resources to maintain them within a collaboration.
Should we try to standardize across HEP?
Should we prefer centralized (e.g. lab-hosted) or decentralized (user-controlled) models?
  \item Software distribution
Spack, UPS/UPD. What do the large experiments use?
CVMFS.
Containers are an important part of job submission.
Some pieces of software are so platform dependent that it's sometimes nearly impossible to replicate the framework on another machine.

  \item Project Management
Fewer users empoy bug-tracking or project management tools. 
Some users use programs such as Trello or Notion for project management or simple Google Docs/Sheets
  \item Documentation
Wikis, Doxygen. Did anyone teach you how to document your code?


\end{itemize}

Discuss Zenodo /Arxiv here? They refer more to data preservation but are collaborative tools.
\subsection{Accessibility issues}
Overlaps with training \& documentation.
How much of “how to do things” lives in private or hard-to-search channels?
How much do these tools enable “chatting with colleagues” and can we improve that?


Experiment analysis software \& environments should be encouraged to guide people towards constructing reproducible and archivable analyses from the start.
Incentives are needed for supporting existing code vs. re-inventing the wheel when unnecessary.

\begin{comment}
Tools for collaboration
git
Many people love git
Some people feel that more git training is needed
\end{comment}


Should this include a discussion on software citations, such as via Zenodo or such?

\subsection{Findings and Recommendations}
The “full” analysis stack of an experiment also includes software that enables interaction between analyzers. This includes documentation of the experiment’s code, messaging between users, discussion forums, software version control, bug tracking, and document workflow management. Especially for small experiments, setting up this infrastructure from scratch can be daunting both in effort and cost.\\
\textit{Recommendation.} Access to a full stack of these services should be provided to funded experiments.  XSEDE/ACES could also play an important role here as some laboratories have onerous access requirements.

\section{Training - Amy}

We do not have the expertise we need for the computational work of upcoming physics.

Question: where can this work come from?

\begin{itemize}
\item Collaborations with industry are rare (are they, though, for GPU work?).  
\item Collaborations with computer scientists are tougher than they appear.  Computer Scientists have a higher pay scale.  Also, much of our work is not interesting from a computer science perspective.
\item We can grow our own.  There are serious barriers to this.  (1) Curriculum change - or at least asking students to double major - is a big ask.  (2) Stable career paths don't currently exist.  Research software engineers are often supported on soft money and computing work is not typically valued in the promotion or tenure process.
\end{itemize}

Training within standard physics curriculum (none)
Training within groups (this burdens groups without access to computing expertise)
Training within collaborations (a good argument for analysis reproducibility is its future usefulness for training)
Training within fields (just getting started, Software Carpentry, Data Carpentry, FIRST-HEP, DANCE-Edu)
Industry training opportunities

\subsection{Findings and Recommendations}
There are many cases where experiments are limited by the availability of personnel trained in various aspects of computing. All students need to be trained in the software and languages used for analysis; beyond that, specialized training may be needed for specific areas, such as GPU use or code optimization. Many traditional avenues (e.g. dedicated academic classes or industry training) may not be a good match to needs or affordable. There are significant efforts through the HEP Software Foundation and other initiatives to develop effective training for HEP software. Software training has a direct connection to concerns about diversity, equity, and inclusion.\\
\textit{Recommendation.} The scalability of various approaches to training should be understood. Best practices from education research, industry, and assessment should be incorporated into efforts. Funding should be made available for the development and hosting of training materials and should include partnerships with education researchers.

\section{Long-Term Preservation}
\subsection{Findings and Recommendations}
Transitioning from the way an analysis is actually done to a “packaged” version that can be rerun by an outsider from scratch is typically complex, as the entire workflow is rarely described in a single place. This causes people to see long-term preservation as an additional burden whose benefits they will not see.\\
\textit{Recommendation.} Provide pipelines to nudge users into choosing practices compatible with long-term preservation as the default. These need to be considered and built in to the structure of analysis systems at the start, not bolted on at the end.

\section{Thoughts}
Must have support mechanisms for both bottom-up and top-down development \& long-term support
%%%%%%%%%%%%%%%%%%%%%%%%%%%%%%%%%%%%%%%%%%

%  If you would like to use BibTEX for the bibliography, please feel free to do so.  It is not required.

%  To use BibTeX,

%    1.  uncomment the following two lines,
%    2.  comment out everything below from  \begin{thebibliography}{99}   to \end{thebibliography).
%    3.  create the file  myreferences.bib in this directory, and process this file in the usual way

\bibliographystyle{JHEP}
\bibliography{Computation/CompF05/myreferences} 

%%%%%%%%%%%%%%%%%%%%%%%%%%%%%%%%%%%%%%%%%


% \begin{thebibliography}{99}

% \input Computation/CompF05/bibliography.tex

% \end{thebibliography}